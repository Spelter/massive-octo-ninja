\section{Old system}
\subsection{Several systems}
Today's system is divided into several semi-independent systems. This is for instance StudentWeb, Innsida 2.0, It's Learning, NTNU's webpages, and so forth. 
StudentWeb is part of bigger system called ``Felles Studentsystem'' (FS) which is NTNU is required to use. This system includes back-office clients and other systems, and changing this
would require years of lobbying and participating in slow working committees to come up with changes. By the time that new system where to be finalized it would probably be plagued with the same problems we see today.

\noindent %Should be same as paragraph{} but without the indent.
 There's also another system which is available for students at the IME-faculty at NTNU, this is called ``EksamensWeb'' and allows students to electronically register complaints to 
 their exams. None of this systems really talk to each other. Data from FS is pushed into It's Learning through batch jobs, but the end result is a set of several, on their own very capable, systems that really doesn't work well together.  
 This lack of interconnection means that a student has to go through several systems to do something simple. If you would like to register into a new course, you would have to deal with three different systems (Figure \pageref{fig:Register-old}). If you would like to complain on a grade on an exam, you must also interact with 3 systems (at best, most students have to mail in a written complaint), some of which demands manual labour from the faculty staff (Figure \pageref{fig:Complain-old}).

\subsection{Complexity}Since the total system is divided into several sub-system, these systems are unnecessary complex and slow. Some of these systems are also:
\begin{itemize}
\item Proprietary closed source.
\item Based on a 15 year old code base.
\item Architecturally locked in to Microsoft-technologies.
\item Expensive to license.
\end{itemize}
This means that they are not easy to upgrade and not very modular. It also means that if we were to upgrade only parts of the current system, it will probably end up just as complex 
as the current system since we would have to consider connections to the parts that are left.


