\section{Background}
In this chapter we will present the background for this project, and review the factors that have
influenced our decisions and vision for this project.

\subsection{Case description}

\begin{quotation}
  \em\noindent NTNU wants to digitize its teaching processes as much as possible.\\\\
   This includes the whole chain from preparation by the teacher to making the curriculum available, lecturing, giving exercises and grading exams/exercises/projects and students complaining about their grades.
\end{quotation}

\subsection{Old system}
Today's system is divided into several semi-independent systems. This is for instance StudentWeb, Innsida 2.0, It's Learning, NTNU's webpages, and so forth. 
StudentWeb is part of bigger system called ``Felles Studentsystem'' (FS) which NTNU is required to use. This system includes back-office clients and other systems, and changing this
would require years of lobbying and participating in slow working committees to come up with changes. By the time that new system where to be finalized it would probably be plagued with the same problems we see today.
~\\\\
%\noindent %Should be same as paragraph{} but without the indent.
 There's also another system which is available for students at the IME-faculty at NTNU, this is called ``EksamensWeb'' and allows students to electronically register complaints to 
 their exams. None of these systems really talk to each other. Data from FS is pushed into It's Learning through batch jobs, and the end result is a set of several systems, which on their own are very capable, but that don't really work well together.  
 This lack of interconnection means that a student has to go through several systems to do something simple. If one would like to register into a new course, one would have to deal with three different systems (Figure \ref{fig:Register-old}). If one would like to complain on a grade given on an exam, one must also interact with three systems (at best, most students have to send in a written complaint), some of which demands manual labour from the faculty staff (Figure \ref{fig:Complain-old}).

\subsubsection{Complexity}Since the total system is divided into several sub-system, these systems are unnecessary complex and slow. Some of these systems are also:
\begin{itemize}
\item Proprietary closed source.
\item Based on a 15 year old code base.
\item Architecturally locked in to Microsoft-technologies.
\item Expensive to license.
\end{itemize}
In our view this means that these systems are very hard to build further upon. It also means that if we were to upgrade only parts of the current system, it will probably end up just as complex as the current system since we would have to consider connections to the parts that are left.

