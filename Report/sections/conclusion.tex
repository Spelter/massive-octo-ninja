\section{Discussion}

So far in this report we've tried to show what we think are some of the big problems with today's solution, and how we would go about to fix it. We are aware that NTNU
has invested a lot of money in It's Learning and that for many different, some of them not even ``political'' reasons NTNU will probably not even consider replacing It's Learning. One 
of the obvious reasons are that enough people actually seem to be content with what they get. As technology students we have higher expectations than many to what a computer system should be, and we do see a lot of shortcomings in the architecture of today's system.\\

\noindent
A lot of the designs made in this project is obviously based on guess-work, and therefore some of our ideas might be more or less impossible to realize. As this has been a relatively small course we've just assumed that what we've designed will work with the existing systems. \\

\noindent
We think that our vision of a new LMS for NTNU should be tailored for NTNU, by NTNU. We want NTNU to own the system as well as the data. It will obviously be a bit of work to 
maintain, but it could foster a nice number of Master thesis' and PhD's, as well as the savings provided by not having to license systems from other providers. The biggest payoff 
would however be that students would be more pleased, more productive and have much better experiences using LMS as a part of their education. If NUDL becomes a success nothing stops 
NTNU from re-selling it to other institutions and over time the system could even pay for itself.\\

\noindent
If made properly NUDL should be more secure than today's solutions, as well as provide the users with a consistent user-experience across platforms and parts of the system. Having a 
system which behaves consistently and looks consistently should be a security factor on it's own by reducing the likelyhood of successfull phishing attacks or insecurity due to wrong 
usage. 


\newpage
\section{Conclusion}
In this report we've shown that today's multitude of systems is a confusing mess for students and staff, hindering them in working effectively and providing little help in day-to-day 
stuff, except digitalizing what would be done manually in older times. Our solution is to remove several of today's systems, and use the rest as subsystems for our new LMS-platform; NUDL. \\

\noindent
NUDL is designed to be built as a modular system with low connectivity and high security. Providing API's allowing faculties to customize how they use it in their work and students to develop tools to simplify their day. Much of the philosophy behind NUDL is shared with many pioneers in open source: We want to provide a rich, free to use, toolset for everyone to utilize. By opening up the system we democraticize the digital learning process. By allowing everyone to participate and chime in (almost) everybody wins. \\

\noindent
Not only is NUDL intended to be an LMS, but we also want it to be a CMS for the whole of NTNU. This way one makes sure that the information that is made public is consistent with the information that isn't. To create NUDL requires much more research and man-hours, but we think it's the only way out of the problems that are inherent with the existing solution.