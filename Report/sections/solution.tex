\section{Our proposed solution}

%sleepy, can somebody proofread?
In this chapter we will discuss our proposed solution to the assignment and how we would like to digitalize NTNU's teaching processes. 

\subsection{What is wrong?}
Having initially looked at todays work processes we see lots of areas where things can be improved. Due to the limitations of this assignment not everything
is being pursued, but the idea behind our proposed solution is that it should be able to address all of the problems with todays solution. 

\noindent
Our biggest complaint about the current solution is that it's very fragmented and requires lots of labour-hours. There are, in our experience, discrepancies between what is 
stated on a webpage describing a subject, and what is the reality. Several of the group members have experienced being told ``we'll update it so it's correct for next year''. 
Staff have to keep their subjects updated in several places, there's course descriptions on NTNU webpages, there's It's Learning, and a lot of courses also have their own separate
web-page for those occasions where It's Learning doesn't cut it, or where they want to make information freely available. 

\noindent
Another problem seems to be the inherent problems with the technologies used. It's Learning tends to not learn a lot, but bugs both the users and itself with un-intuitive behaviour, 
odd bugs and browser incompatibility. For the last months logging into It's Learning has been messed up because it ends up just redirecting itself back to Innsida where one has to 
click once more. It's Learning is for the most part a standalone system in a continously more interconnected web full of ``open'' standards. The way we see it, It's Learning is the 
rod that has been pushed between the spokes of the wheel. There's a lot wrong with the system, but it's not all bad. It's just not right. In 2013 It's Learning is sadly holding the 
rest back. 

\subsection{What do we do?}
% http://www.ntnu.no/c/document_library/get_file?uuid=cc8a29fa-84f4-44b3-9af4-36e6c486746c&groupId=524136
% http://www.ntnu.no/c/document_library/get_file?uuid=ce1ba8c9-6744-42ff-bfe7-c114d58a36a6&groupId=524136
% file:///D:/Nedlastinger/Nettleser/hva_vet_vi_om_bruk_av_lms_i_uh-sektoren_4.ppt < "Evalueringer av LMS"
Based on our and others analysis we conclude that it's time to replace It's Learning. On the long term our plan is not only to replace It's Learning, but to create
a CMS and LMS system that envelopes the whole of NTNU. It does sound very ambitious, luckily the plan is not to create a huge blob of software, but rather a modularized tiered 
platform where data is stored in a centralized storage. The idea is to create a secure, robust, and manageable platform which is testable, modularized, and compartmentalized.
We also want to implement one thing that we really feel is missing today - search. 

\noindent
Do you remember what was before Google? Yes, obviously you do, but do you remember how it really felt before Google? Maybe not so much. Search is good, but good search is better. 
According to Google Zeitgeist (search statistics) \cite{google:zeitgeist} people are to lazy to type in facebook.com, they just search for "facebook". Granted, the average NTNU 
student isn't the average person, but why should students have to manually navigate complex navigation trees when they can type a couple of words and get the result they want 
immediately? We think that by designing our system architecture and data storage with search in mind, we can bring value and help both students and staff to save time. 

\noindent
Today students have to interact with studentweb and It's Learning. Data gets passed one way only. We want our system to bridge the gap between the two by using a 
connector-pattern %lack of better name at the moment, can we come up with the right name for this pattern?
to link our system and FS/StudentWeb. This way, if one changes it's easy to adjust, and there is room for making our system better while the communication with FS is still bound by 
the same contract. We want to make our system interact with StudentWeb on part of the student. That means that the students interact with a system that is consistent. As a security 
measure our system will log error messages and calls, and in rare cases where propagation does not work as expected alerts can be raised and personel with the right clearances can 
take action and manually merge the discrepancy. %not sure about this last part(?)

\subsection{How do we do it}

